\input{wsc14style.tex}

\documentclass{wscpaperproc}
\usepackage{latexsym}
%\usepackage{caption}
\usepackage{graphicx}
\usepackage{mathptmx}
\usepackage[utf8]{inputenc}

\usepackage{amsmath}
\usepackage{amsfonts}
\usepackage{amssymb}
\usepackage{amsbsy}
\usepackage{amsthm}

\usepackage[pdftex,colorlinks=true,urlcolor=blue,citecolor=black,anchorcolor=black,linkcolor=black]{hyperref}

% If you use theoremes
\newtheoremstyle{wsc}% hnamei
{3pt}% hSpace abovei
{3pt}% hSpace belowi
{}% hBody fonti
{}% hIndent amounti1
{\bf}% hTheorem head fontbf
{}% hPunctuation after theorem headi
{.5em}% hSpace after theorem headi2
{}% hTheorem head spec (can be left empty, meaning `normal')i

\theoremstyle{wsc}
\newtheorem{theorem}{Theorem}
\renewcommand{\thetheorem}{ \arabic{theorem}}
\newtheorem{corollary}[theorem]{Corollary}
\renewcommand{\thecorollary}{\arabic{corollary}}
\newtheorem{definition}{Definition}
\renewcommand{\thedefinition}{\arabic{definition}}

\begin{document}

\WSCpagesetup{Ortega and Salgado}

% AUTHOR: Enter the title, all letters in upper case
\title{Farmers, Bandits and Soldiers an agent based simulation about delinquency in the
       society}

\author{Andres Felipe Ortega Montoya\\ [12pt]
aortega7@eafit.edu.co\\
Ingenieria Matemática\\
Universidad EAFIT\\
\and
Alejandro Salgado Gómez\\[12pt]
asalgad2@eafit.edu.co\\
Ingeniería Matemática\\
Universidad EAFIT\\
}

\maketitle

\section*{ABSTRACT}
Delincuency is one of the main probles that societies face nowadays,
one the main reasons of this is the burden that it represents in the
economy and the effect that has in the life quality of the population.
This document describes the implementation of an agent base model that aims to
simulate a simplified version of a society and some of their interactions,
specifically those who have to do with the effects of delinquency.
In this work the society is conformed by three social groups, the
first one are farmers, that represent the productive class, the scond
one are bandits, which represents the delinquency, and the last ones
are soldiers, that represents the public control. This work takes as base point
the ideas proposed in some articles of systems dynamics and adapt them to an
agent based simulation. As implementation tool we use the NetLogo software
\cite{netlogo}.

\section{Introduction}

Delinquency affects a lot of countries all arround the world. Because of the
continuous problems that this actions generate in terms of social, econmical and
political aspects, the nations continue to spend huge amount of resources
trying to keep this problematic as controled as possible, but the results achived
are very poor, as seen in the indices of peace. Actually countries have become
less peaceful during the last decade, bringing enourmous concequences to the
society. \cite{peace} \cite{violence}

With this motivation the authors developed an agent based model in which they
represent an artificial society that is conformed by three social groups,
Farmers, Bandits and Soldiers. The main objective of this model is to
analyse the effect of delincuency by testing different scenarios that
occur based on the configuration of the parameters of the simulation,
in order to see how this phenomenon affects the society and its
economy.

In the following sections we present some previous work that has been developed
in this topic, later a detailed description of the model its given, with the
help of the ODD (Overview, Design concepts and Details) methodology \cite{odd},
then an implementation section is introduced, there the details of the creation 
process of the model will be explained. Lastly a section about the experiments
conducted with the model is shown and the conclusions of the work are described
at the end of the document.

\section{Previous work}

This work is based on \cite{article1} \cite{article2}

\section{Conceptual modeling}

\subsection{Generalities}

This model is the representation of a simplified society in which three types
of building define the infractucture necessary for the interactions that occurs
between the three social groups. In this model the
economy is based on a single type of resource which represents all commodities,
this resorces are produced by farmers in the first type of building named
farms, this buildings are assume as an infite source of income. This
resources are also used as currency to pay taxes in the second type of
building, named city hall, this building reseprent the government
institutions and store the supplys that are used to sustain soldiers,
which are in charge of seeking for the well being of the farmers by
keeping bandit population at bay. The last building type, named houses,
are used by farmers and bandits to store the spare resources that they
earn either by gathering them in a farm, in the case of farmers, or
stealing, in the case of bandits.

\subsubsection{Purpose}

The purpose of the model is to analyse how delincuency affects the normal
development of a socity, from the effect that has in the workforce and how
reactions, like incrementing the public force angets, emerge as responce to
protect citizens from criminality.

\subsubsection{Entities, state variables and scales}

\noindent \textbf{Entities}

\begin{itemize}
    \item Bandits: Steal resources from farmers to sustain themselves while trying to avoid soldiers.
    \item Farmers: Gather food to sustain themselves and the soldiers by paying taxes.
    \item Soldiers: Invest resources in the search of bandits to cpture.
    \item Buildings: Store agent's resources for latter consumption.
\end{itemize}

\noindent \textbf{State variables}

Human agents

\begin{itemize}
    \item Energy
    \item Capacity
    \item Occupation
\end{itemize}

Buildings

\begin{itemize}
    \item Storage
\end{itemize}

\noindent \textbf{Scales}

\begin{itemize}
    \item Ticks means 6 min
    \item Patch means 1 Km squared
\end{itemize}

\subsubsection{Process overview and scheduling}

\begin{itemize}
    \item move
    \item collect
    \item switch-occupation
\end{itemize}

On each tick all the agents either move or collect depending on the pathc that
they are and the energy available, then if the conditions are not favorable for
the agent then, based on a fixed probability, they choose to continue in their
current occupation or switch to another one.

\subsection{Design concepts}

\subsubsection{Basic principles}

All agents want to collect resources, but not in a standard way. There is a
finite amount of weight and energy that each individual can have.

\subsubsection{Emergence}

The behaviour that emerge from the model are the agglomeration of agents in
some parts of the map

\subsubsection{Adaptation}

The agents adapt to their current situation by means of changing their
ocupation if the conditions are unfavorable.

\subsubsection{Objectives}

The objective of the farmers is to collect as much resources as they can, in
the case of bandits the objective is to steal such resources from farmers, and
the soldiers seek to decreace the amount of bandits in the society. This
objectives are affected by each individual's state variables.

\subsubsection{Learning}

Learning capabilities of agents are not taken into account in this model

\subsubsection{Prediction}

Prediction procedures are not taken into account as part of the agent's
behaviour in this model

\subsubsection{Sensing}

Each agent can see other agents and can decide how to act based on this
knowledge.

\subsubsection{Interaction}

Bandits assoult farmers thus stealing all the resources they are currently
carrying. Soldiers capture bandits and try to convert them into farmers.

\subsubsection{Stochasticity}

The process in which an agent changes occupation is random, which implies that
each execution can not be reproduces exactly.

\subsubsection{Collectives}

Agents are divided in three groups that defines the behaviour they have to
follow and the way they are going to seek their objectives

\subsubsection{Observation}

The variables that are necessary to know the current state of the system are
the population of each social group, the violence indicator and the capital
flow.

\subsection{Details}

\subsubsection{Initialization}

The position of all agents are chosen randomly, the amount of subjects per
social group is recived as parameter to the model as well as the probabilities
of switching current faction.

\subsubsection{Input}

\subsubsection{Submodels}

Farmers move towards the nearest farm or city hall depending on the weight they
are carrying. but if their energy is sufficiently low they will decide to go
home to refill it. Bandits move towards the nearest farmer trying to assoult
them, but also trying to avoid soldiers, finally if their energy is too low
they will return to their home to refill it. Soldiers move towards the nearest
bandit in their field of view trying to convert them into farmers, if the
energy they have is too low they will move towards the nearest city hall to
refill it

\bibliographystyle{wsc}
\bibliography{entrega1}

\end{document}
