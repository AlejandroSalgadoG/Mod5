\input{wsc14style.tex}

\documentclass{wscpaperproc}
\usepackage{latexsym}
%\usepackage{caption}
\usepackage{graphicx}
\usepackage{mathptmx}
\usepackage[utf8]{inputenc}

\usepackage{amsmath}
\usepackage{amsfonts}
\usepackage{amssymb}
\usepackage{amsbsy}
\usepackage{amsthm}

\usepackage[pdftex,colorlinks=true,urlcolor=blue,citecolor=black,anchorcolor=black,linkcolor=black]{hyperref}

% If you use theoremes
\newtheoremstyle{wsc}% hnamei
{3pt}% hSpace abovei
{3pt}% hSpace belowi
{}% hBody fonti
{}% hIndent amounti1
{\bf}% hTheorem head fontbf
{}% hPunctuation after theorem headi
{.5em}% hSpace after theorem headi2
{}% hTheorem head spec (can be left empty, meaning `normal')i

\theoremstyle{wsc}
\newtheorem{theorem}{Theorem}
\renewcommand{\thetheorem}{ \arabic{theorem}}
\newtheorem{corollary}[theorem]{Corollary}
\renewcommand{\thecorollary}{\arabic{corollary}}
\newtheorem{definition}{Definition}
\renewcommand{\thedefinition}{\arabic{definition}}

\begin{document}

\WSCpagesetup{Ortega and Salgado}

% AUTHOR: Enter the title, all letters in upper case
\title{Farmers, Bandits and Soldiers an agent based simulation}

\author{Andres Felipe Ortega Montoya\\ [12pt]
aortega7@eafit.edu.co\\
Ingenieria Matemática\\
Universidad EAFIT\\
\and
Alejandro Salgado Gómez\\[12pt]
asalgad2@eafit.edu.co\\
Ingeniería Matemática\\
Universidad EAFIT\\
}

\maketitle

\section*{ABSTRACT}
This is the abstract

\section{INTRODUCTION}

This is the introduction

\section{Purpose}

The purpose of the model is to nalyse the behaviour of an artificial society
populated by bandits, farmers and soldiers basedon policies that are taken as
parameters to the model.

\section{Entities, state variables and scales}

\subsection{Entities}

\begin{itemize}
    \item Bandits: Steal resources from farmers to sustain themselves while trying to avoid soldiers.
    \item Farmers: Gather food to sustain themselves and the soldiers by paying taxes.
    \item Soldiers: Invest resources in the search of bandits to cpture.
    \item Buildings: Store agent's resources for latter consumption.
\end{itemize}

\subsection{State variables}

Human agents

\begin{itemize}
    \item Energy
    \item Capacity
    \item Occupation
\end{itemize}

Buildings

\begin{itemize}
    \item Storage
\end{itemize}

\subsection{Scales}

\begin{itemize}
    \item Ticks means 6 min
    \item Patch means 1 Km squared
\end{itemize}

\section{Process overview and scheduling}

\begin{itemize}
    \item move
    \item collect
    \item switch-occupation
\end{itemize}

On each tick all the agents either move or collect depending on the pathc that
they are and the energy available, then if the conditions are not favorable for
the agent then, based on a fixed probability, they choose to continue in their
current occupation or switch to another one.

\section{Design concepts}

\subsection{Basic principles}

All agents want to collect resources, but not in a standard way. There is a
finite amount of weight and energy that each individual can have.

\subsection{Emergence}

The behaviour that emerge from the model are the agglomeration of agents in
some parts of the map

\subsection{Adaptation}

The agents adapt to their current situation by means of changing their
ocupation if the conditions are unfavorable.

\subsection{Objectives}

The objective of the farmers is to collect as much resources as they can, in
the case of bandits the objective is to steal such resources from farmers, and
the soldiers seek to decreace the amount of bandits in the society. This
objectives are affected by each individual's state variables.

\subsection{Learning}

Learning capabilities of agents are not taken into account in this model

\subsection{Prediction}

Prediction procedures are not taken into account as part of the agent's
behaviour in this model

\subsection{Sensing}

Each agent can see other agents and can decide how to act based on this
knowledge.

\subsection{Interaction}

Bandits assoult farmers thus stealing all the resources they are currently
carrying. Soldiers capture bandits and try to convert them into farmers.

\subsection{Stochasticity}

The process in which an agent changes occupation is random, which implies that
each execution can not be reproduces exactly.

\subsection{Collectives}

Agents are divided in three groups that defines the behaviour they have to
follow and the way they are going to seek their objectives

\subsection{Observation}

The variables that are necessary to know the current state of the system are
the population of each social group, the violence indicator and the capital
flow.

\section{Details}

\subsection{Initialization}

The position of all agents are chosen randomly, the amount of subjects per
social group is recived as parameter to the model as well as the probabilities
of switching current faction.

\subsection{Submodels}

Farmers move towards the nearest farm or city hall depending on the weight they
are carrying. but if their energy is sufficiently low they will decide to go
home to refill it. Bandits move towards the nearest farmer trying to assoult
them, but also trying to avoid soldiers, finally if their energy is too low
they will return to their home to refill it. Soldiers move towards the nearest
bandit in their field of view trying to convert them into farmers, if the
energy they have is too low they will move towards the nearest city hall to
refill it

\section{Citation}

The format for other types of reference can be inferred from the examples in the references, which include:
\begin{itemize}
\item a technical report \cite{chi89},
\item a proceedings article \cite{cheng:input94},
\item a journal article \cite{gupta:mnormal},
\item a book by 2 authors \cite{hammersley:montecarlo},
\item a chapter in a book \cite{sch79},
\item an unpublished thesis or dissertation \cite{ste99},
\item a book with no identified authors \cite{chicago03}, and
\item a document available on the web \cite{Foundation}.
\end{itemize}

\bibliographystyle{wsc}
\bibliography{entrega1}

\section*{AUTHOR BIOGRAPHIES}

\noindent {\bf ANDRES FELIPE ORTEGA MONTOYA} His email address is \email{aortega7@eafit.edu.co}.\\
\noindent {\bf ALEJANDRO SALGADO GÓMEZ} His email address is \email{asalgad2@eafit.edu.co}.\\

\end{document}
