\input{wsc14style.tex}

\documentclass{wscpaperproc}
\usepackage{latexsym}
%\usepackage{caption}
\usepackage{graphicx}
\usepackage{mathptmx}
\usepackage[utf8]{inputenc}

\usepackage{amsmath}
\usepackage{amsfonts}
\usepackage{amssymb}
\usepackage{amsbsy}
\usepackage{amsthm}

\usepackage[pdftex,colorlinks=true,urlcolor=blue,citecolor=black,anchorcolor=black,linkcolor=black]{hyperref}

% If you use theoremes
\newtheoremstyle{wsc}% hnamei
{3pt}% hSpace abovei
{3pt}% hSpace belowi
{}% hBody fonti
{}% hIndent amounti1
{\bf}% hTheorem head fontbf
{}% hPunctuation after theorem headi
{.5em}% hSpace after theorem headi2
{}% hTheorem head spec (can be left empty, meaning `normal')i

\theoremstyle{wsc}
\newtheorem{theorem}{Theorem}
\renewcommand{\thetheorem}{ \arabic{theorem}}
\newtheorem{corollary}[theorem]{Corollary}
\renewcommand{\thecorollary}{\arabic{corollary}}
\newtheorem{definition}{Definition}
\renewcommand{\thedefinition}{\arabic{definition}}

\begin{document}

\WSCpagesetup{Ortega and Salgado}

% AUTHOR: Enter the title, all letters in upper case
\title{Farmers, Bandits and Soldiers an agent based simulation about delinquency in the
       society}

\author{Andres Felipe Ortega Montoya\\ [12pt]
aortega7@eafit.edu.co\\
Ingenieria Matemática\\
Universidad EAFIT\\
\and
Alejandro Salgado Gómez\\[12pt]
asalgad2@eafit.edu.co\\
Ingeniería Matemática\\
Universidad EAFIT\\
}

\maketitle

\section*{ABSTRACT}
Delincuency is one of the main probles that societies face nowadays,
one the main reasons of this is the burden that it represents in the
economy and the effect that has in the life quality of the population.
This document describes the implementation of an agent base model that aims to
simulate a simplified version of a society and some of their interactions,
specifically those who have to do with the effects of delinquency.
In this work the society is conformed by three social groups, the
first one are farmers, that represent the productive class, the scond
one are bandits, which represents the delinquency, and the last ones
are soldiers, that represents the public control. This work takes as base point
the ideas proposed in some articles of systems dynamics and adapt them to an
agent based simulation. As implementation tool we use the NetLogo software
\cite{netlogo}.

\section{Introduction}

Delinquency affects a lot of countries all arround the world. Because of the
continuous problems that this actions generate in terms of social, econmical and
political aspects, the nations continue to spend huge amount of resources
trying to keep this problematic as controled as possible, but the results achived
are poor, as seen in the indices of peace. Actually countries have become
less peaceful during the last decade, bringing enourmous concequences to the
society. \cite{peace} \cite{violence}

With this motivation the authors developed an agent based model in which they
represent an artificial society that is conformed by three social groups,
Farmers, Bandits and Soldiers. The main objective of this model is to
analyse the effect of delincuency by testing different scenarios that
occur based on the configuration of the parameters of the simulation,
in order to see how this phenomenon affects the society and its
economy.

In the following sections we present some previous work that has been developed
in this topic, later a detailed description of the model its given, with the
help of the ODD (Overview, Design concepts and Details) methodology \cite{odd},
then an implementation section is introduced, there the details of the creation
process of the model will be explained. Lastly a section about the experiments
conducted with the model is shown and the conclusions of the work are described
at the end of the document.

\section{Previous work}

This work is based on two main acticles, both addressed the problem from the
system dynamics perspective, however the authors find that the ideas proposed in
this articles can be applied in an agent base simulation, giving the possibility
to explore new caracteristics of the problem and see others in a more
detail way, thanks to the granularity that this type of modelation offers.

The first article studies an artificial society where they seek for politics
that leads to a state of peace. In that work they represent the society with
the same groups as in the present work, and state that the members of the
society can change their role, based on psychological or economical conditions,
such as the level of empathy with their social group or the profits
that their occupations generate. The objective was to find politics
that leads to an estabilization of the system in a state of peace and
prosperity by changing different parameters of the model, such as the
sensivility to violence or the initial population of the different social
groups. \cite{article1}

The second article is about the confrontations between an insurgent force and
militar forces of a government. In that work they describe the effects and
concequencies about different strategies that can be used to try to solve the
problem, always with the goal of minimizing the threats civilians are exposed to.
This article states that one of the main factors that has to be taken
into account is the support that the population has to its government, because
this is the fact that defines if the civilian population will became part of the
insurgents rather than help the government, because it is in disagreement with
the actual state of the society. \cite{article2}

\section{Conceptual modeling}

\subsection{Generalities}

This model is the representation of a simplified society in which three types
of building define the infractucture necessary for the interactions that occurs
between the three social groups. In this model the
economy is based on a single type of resource which represents all commodities,
this resorces are produced by farmers in the first type of building named
farms, this buildings are assume as an infite source of income. This
resources are also used as currency to pay taxes in the second type of
building, named city hall, this building reseprent the government
institutions and store the supplys that are used to sustain soldiers,
which are in charge of seeking for the well being of the farmers by
keeping bandit population at bay. The last building type, named houses,
are used by farmers and bandits to store the spare resources that they
earn either by gathering them in a farm, in the case of farmers, or
stealing, in the case of bandits.

\subsubsection{Purpose}

The purpose of the model is to analyse how delincuency affects the normal
development of a socity, from the effect that has in the workforce and how
reactions, like incrementing the public force angets, emerge as responce to
protect citizens from criminality.

\subsubsection{Entities and state variables}
\noindent \textbf{Entities}

\begin{itemize}
    \item Farms: Provides an endless source of the necessary resources for the
    population to subsist with.

    \item Houses: Serves as an storage for the spare resources an indivial has
    after using a portion of those to meet its needs. If a given individual
    does not carry enought resources to refill its energy, they can use the
    ones they have previously stored in their house.

    \item City Hall: It stores the resources that farmers pay as taxes and also
    the ones soldiers seize from bandits. Soldiers can use the funds in the
    city hall to sustain themselves.

    \item Farmers: They gather resources from farms and pay a certain amount of
    those resources in the city hall as a way of paying soldiers for their
    protection.

    \item Bandits: Steal resources from farmers to sustain themselves while
    trying to avoid soldiers.

    \item Soldiers: Pursue bandits in order to reintegrate them into society as
    farmers and recover the funds they have stolen.
\end{itemize}

\noindent \textbf{State variables}\\

\noindent The common atributes farmers, bandits and soldiers have are:

\begin{itemize}
    \item Energy: This serves as a way of telling of how much time an agent
    have before it needs to eat and thus return to its home or in the case or
    soldiers, city hall.

    \item Load: It reprecents the quantity of resources a given agent is
    carrying with itself at the moment.

    \item Destination: This is the current destination a given agent seeks to
    move towards to. In the case of farmers this can be their house, farm or
    city hall. In the case of bandits this can be a possible farmer to rob or
    their house. And finally it can be a nearby bandit or their city hall in
    the case of soldiers.
\end{itemize}

\noindent Houses and City Halls share a common atribute named inventory which
is the amount of resources it is storing at the moment.\hfill\break

\subsubsection{Process overview and scheduling}

\begin{itemize}
    \item Move
    \item Work
    \item Pay-taxes
    \item Steal
    \item Seize
    \item Rest
    \item Switch-occupation
\end{itemize}

On each tick all agents will lose 1 unit of energy and will move to their
destination, if they have zero energy then they will start consuming the
resources they are carrying at the moment to supply their lack of energy.

In the case an agent is a farmer it will cicle through their assigned farm,
city hall and house.  When they get to a farm they will work and gather as
much resources as they can carry.  When arriving to a city hall then they
will pay a fixed amount of the resources they are carrying as taxes, such
resources are then stored in the city hall.  After they get to their house a
they will rest and consume as much of the resources they have at their
disposal to refill their energy, if they do not have enough of them to do
so, they will change their current occupation to be a bandit or a soldier
based on a given probability. On the other hand, if they have more resources
than needed they will store the surplus in their house for future use.

If the agent is a bandit it will chase after the farmer with non-zero load that
is nearest to them and if they get close enough they will take all of their
resourses. In the case there is a nearby soldier they will run away from it,
otherwise they will seek a new farmer to steal from until they either reach
the maximum load they are able to carry or they empty their energy to
return home. When they get home they will do in the same manner as farmers
do, the only difference being that they can only switch occupation to
farmer if the situation arises.

Finally, soldiers chase after their nearest bandit in order to seize their
belongings and convert them into farmers. In the case their energy hits zero,
they will return to their city hall to rest and recover their energy
by consuming the resources available in the building. In the
situation that there are not enough resources to meet their needs
they will become farmers.

\subsection{Design concepts}

\subsubsection{Basic principles}

In this model it is supposed that the population is constant over time.  All
agents want to collect resources, each one in a different way depending on its
occupation. Also it is considered that there is a finite amount of weight and
energy that each individual can have. The spupply of resources available on
farms and the amount of resources that can be stored in a building is taken
as infinite. Finally each person has their own house.

\subsubsection{Emergence}

There are many emergence behaviour in the model like the agglomeration of
agents in some parts of the map as a consequence of various farmers being
assigned to the same workplace. This attracts bandits thus incrementing the
criminality nearby this places and therefore making soldiers to appear in this
zones. Also the appearrance of sectors with a high dencity of bandits living in
it due to the remoteness that some places have from farms or city halls, making
the living as a farmer unsustainable.

\subsubsection{Adaptation}

In the model agents adapt to their current situation if the conditions does not
provides enough resources to supply for a living by means of changing their
ocupation. It is important to notice that the social groups that an agent can
switch to depends on the occupation that they are currently performing.

\subsubsection{Objectives}

The objective of an agent varies depending on their current occupation. Farmers
try to collect as much resources as they can by working their respective farms.
Bandits try to steal as much resources from farmers as they can, trying not to
be captured by soldiers. Finally soldiers seek to decreace the amount of
bandits in the society.

\subsubsection{Learning}

Learning capabilities of agents are not taken into account in this model.

\subsubsection{Prediction}

Prediction procedures are not taken into account as part of the agent's
behaviour in this model.

\subsubsection{Sensing}

Each agent can see other agents and can decide how to act based on this
knowledge.

\subsubsection{Interaction}

Bandits assoult farmers thus stealing all the resources they are currently
carrying. Soldiers capture bandits and try to convert them into farmers.

\subsubsection{Stochasticity}

The process in which an agent changes occupation is random, which implies that
each execution can not be reproduces exactly.

\subsubsection{Collectives}

Agents are divided in three groups that defines the behaviour they have to
follow and the way they are going to seek their objectives

\subsubsection{Observation}

The variables that are necessary to know the current state of the system are
the population of each social group, the violence indicator and the capital
flow.

\subsection{Details}

\subsubsection{Initialization}

The position of all agents are chosen randomly, the amount of subjects per
social group is recived as parameter to the model as well as the probabilities
of switching current faction.

\subsubsection{Input}

\subsubsection{Submodels}

Farmers move towards the nearest farm or city hall depending on the weight they
are carrying. but if their energy is sufficiently low they will decide to go
home to refill it. Bandits move towards the nearest farmer trying to assoult
them, but also trying to avoid soldiers, finally if their energy is too low
they will return to their home to refill it. Soldiers move towards the nearest
bandit in their field of view trying to convert them into farmers, if the
energy they have is too low they will move towards the nearest city hall to
refill it

\bibliographystyle{wsc}
\bibliography{entrega1}

\end{document}
